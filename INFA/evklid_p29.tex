\documentclass[10pt, onecolumn]{article}

\usepackage{tikz}
\usepackage{tkz-euclide}
\usetikzlibrary{angles,quotes}
\usepackage[english,russian]{babel}
\usepackage[utf8]{inputenc}
\usepackage[T2A]{fontenc}
\usepackage[left=5mm, top=20mm, right=5mm, bottom=10mm, nofoot]{geometry}
\usepackage{tempora}
\usepackage{newtxmath}
\usepackage{multicol}
\usepackage{graphicx}

\date{}

\setlength{\columnsep}{-3cm}

\begin{document}

    % Команды с отрисовкой, чтобы не дублировать код
    \newcommand{\AB}{
        \begin{tikzpicture}
        
            \draw[line width=1mm, red] (0, 0) -- (1, 0);
            \filldraw[red] (0, 0) circle (1pt) node[scale=0.5, anchor=south] {A};
            \filldraw[red] (1, 0) circle (1pt) node[scale=0.5, anchor=south] {B};
            
        \end{tikzpicture}
    }
    
    \newcommand{\AC}{
        \begin{tikzpicture}
        
            \draw[line width=1mm, red] (0, 0) -- (1, 0);
            \filldraw[red] (0, 0) circle (1pt) node[scale=0.5, anchor=south] {A};
            \filldraw[red] (1, 0) circle (1pt) node[scale=0.5, anchor=south] {C};
            
        \end{tikzpicture}
    }

    \newcommand{\BD}{
        \begin{tikzpicture}
        
            \draw[line width=1mm, yellow] (0, 0) -- (1, 0);
            \filldraw[yellow] (0, 0) circle (1pt) node[scale=0.5, anchor=south] {B};
            \filldraw[yellow] (1, 0) circle (1pt) node[scale=0.5, anchor=south] {D};
            
        \end{tikzpicture}
    }

    \newcommand{\CE}{
        \begin{tikzpicture}
        
            \draw[line width=1mm, yellow] (0, 0) -- (1, 0);
            \filldraw[yellow] (0, 0) circle (1pt) node[scale=0.5, anchor=south] {C};
            \filldraw[yellow] (1, 0) circle (1pt) node[scale=0.5, anchor=south] {E};
            
        \end{tikzpicture}
    }

     \newcommand{\BE}{
        \begin{tikzpicture}
        
            \draw[line width=1mm, blue] (0, 0) -- (1, 0);
            \filldraw[blue] (0, 0) circle (1pt) node[scale=0.5, anchor=south] {B};
            \filldraw[blue] (1, 0) circle (1pt) node[scale=0.5, anchor=south] {E};
            
        \end{tikzpicture}
    }
    
    \newcommand{\CD}{
        \begin{tikzpicture}
        
            \draw[line width=1mm, blue] (0, 0) -- (1, 0);
            \filldraw[blue] (0, 0) circle (1pt) node[scale=0.5, anchor=south] {C};
            \filldraw[blue] (1, 0) circle (1pt) node[scale=0.5, anchor=south] {D};
            
        \end{tikzpicture}
    }

    \newcommand{\AD}{
        \begin{tikzpicture}
        
            \draw[line width=1mm, red] (0, 0) -- (0.5, 0);
            \draw[line width=1mm, yellow] (0.5, 0) -- (1, 0);

            \filldraw[red] (0, 0) circle (1pt) node[scale=0.5, anchor=south] {A};
            \filldraw[yellow] (1, 0) circle (1pt) node[scale=0.5, anchor=south] {D};
            
        \end{tikzpicture}
    }

    \newcommand{\EA}{
        \begin{tikzpicture}
        
            \draw[line width=1mm, red] (0, 0) -- (0.5, 0);
            \draw[line width=1mm, yellow] (0.5, 0) -- (1, 0);

            \filldraw[red] (0, 0) circle (1pt) node[scale=0.5, anchor=south] {A};
            \filldraw[yellow] (1, 0) circle (1pt) node[scale=0.5, anchor=south] {E};
            
        \end{tikzpicture}
    }

    \newcommand{\ABCblack}{
        \begin{tikzpicture}[scale=0.5]
        \coordinate (A) at (1, 0);
        \coordinate (C) at (0, -2);
        \coordinate (B) at (2, -2);
        
        \draw pic[fill=black, angle radius=1cm] {angle = C--A--B};
        
        \draw node[scale=0.7, anchor=north] at (1, 1) {A};
        \draw node[scale=0.7, anchor=south] at (0, -3) {C};
        \draw node[scale=0.7, anchor=south] at (2, -3) {B};
        
        
    \end{tikzpicture}

    }

    \newcommand{\ABE}{
        \begin{tikzpicture}[scale=0.5]
        \coordinate (A) at (1, 0);
        \coordinate (C) at (0, -2);
        \coordinate (B) at (2, -2);
        \coordinate (E) at (-1, -4);
        
        \draw pic[fill=black, angle radius=1cm] {angle = C--A--B};
        
        \draw[line width=1mm, red] (1, 0) -- (0, -2);
        \draw[line width=1mm, red] (1, 0) -- (2, -2);
        \draw[line width=1mm, yellow] (0, -2) -- (-1, -4);
        \draw[line width=1mm, blue] (2, -2) -- (-1, -4);

        \draw node[scale=0.7, anchor=north] at (1, 1) {A};
        \draw node[scale=0.7, anchor=south] at (2, -3) {B};
        \draw node[scale=0.7, anchor=south] at (-1.5, -5) {E};

        
    \end{tikzpicture}

    }

    \newcommand{\ACD}{
        \begin{tikzpicture}[scale=0.5]
        \coordinate (A) at (1, 0);
        \coordinate (C) at (0, -2);
        \coordinate (B) at (2, -2);
        \coordinate (D) at (2, 4);
        
        \draw pic[fill=black, angle radius=1cm] {angle = C--A--B};
        
        \draw[line width=1mm, red] (1, 0) -- (0, -2);
        \draw[line width=1mm, red] (1, 0) -- (2, -2);
        \draw[line width=1mm, yellow] (2, -2) -- (3, -4);
        \draw[line width=1mm, blue] (0, -2) -- (3, -4);

        \draw node[scale=0.7, anchor=north] at (1, 1) {A};
        \draw node[scale=0.7, anchor=south] at (-0.5, -2.5) {C};
        \draw node[scale=0.7, anchor=south] at (3.5, -5) {D};

        
    \end{tikzpicture}

    }

    \newcommand{\ABEarea}{
        \begin{tikzpicture}[scale=0.5]
        \coordinate (A) at (1, 0);
        \coordinate (C) at (0, -2);
        \coordinate (B) at (2, -2);
        \coordinate (E) at (-1, -4);

        \draw pic[fill=yellow, angle radius=1cm] {angle = A--B--E};
        \draw pic[fill=blue, angle radius=1cm] {angle = A--B--C};
        

        \draw node[scale=0.7, anchor=north] at (1, 1) {A};
        \draw node[scale=0.7, anchor=south] at (2.5, -3) {B};
        \draw node[scale=0.7, anchor=south] at (0, -4) {E};

        
    \end{tikzpicture}

    }

    \newcommand{\ACDarea}{
        \begin{tikzpicture}[scale=0.5]
        \coordinate (A) at (1, 0);
        \coordinate (C) at (0, -2);
        \coordinate (B) at (2, -2);
        \coordinate (D) at (3.5, -5);

        \draw pic[fill=yellow, angle radius=1cm] {angle = D--C--A};
        \draw pic[fill=blue, angle radius=1cm] {angle = B--C--A};
        
        
        \draw node[scale=0.7, anchor=north] at (1, 1) {A};
        \draw node[scale=0.7, anchor=south] at (-0.5, -2.5) {C};
        \draw node[scale=0.7, anchor=south] at (2, -4) {D};

        
    \end{tikzpicture}

    }

    \newcommand{\ABCblueright}{
        \begin{tikzpicture}[scale=0.5]
        \coordinate (A) at (1, 0);
        \coordinate (C) at (0, -2);
        \coordinate (B) at (2, -2);

        \draw pic[fill=blue, angle radius=1cm] {angle = A--B--C};
        

        \draw node[scale=0.7, anchor=north] at (1, 1) {A};
        \draw node[scale=0.7, anchor=south] at (2.5, -3) {B};
        \draw node[scale=0.7, anchor=south] at (-0.5, -3) {C};

        
    \end{tikzpicture}

    }

    \newcommand{\ABCblueleft}{
        \begin{tikzpicture}[scale=0.5]
        \coordinate (A) at (1, 0);
        \coordinate (C) at (0, -2);
        \coordinate (B) at (2, -2);

        \draw pic[fill=blue, angle radius=1cm] {angle = B--C--A};
        

        \draw node[scale=0.7, anchor=north] at (1, 1) {A};
        \draw node[scale=0.7, anchor=south] at (2.5, -3) {B};
        \draw node[scale=0.7, anchor=south] at (-0.5, -3) {C};

        
    \end{tikzpicture}

    }

    \newcommand{\BCEblueleft}{
        \begin{tikzpicture}[scale=0.5]
        \coordinate (C) at (1, 0);
        \coordinate (E) at (0, -2);
        \coordinate (B) at (2, -1);

        \draw pic[fill=red, angle radius=1cm] {angle = B--E--C};
        

        \draw node[scale=0.7, anchor=north] at (1, 1) {C};
        \draw node[scale=0.7, anchor=south] at (2.5, -2) {B};
        \draw node[scale=0.7, anchor=south] at (-0.5, -3) {E};

        
    \end{tikzpicture}

    }

    \newcommand{\BCDblueright}{
        \begin{tikzpicture}[scale=0.5]
        \coordinate (C) at (-2, 3);
        \coordinate (D) at (0, 2);
        \coordinate (B) at (-1, 4);

        \draw pic[fill=red, angle radius=1cm] {angle = B--D--C};
        

        \draw node[scale=0.7, anchor=north] at (-2, 3) {C};
        \draw node[scale=0.7, anchor=south] at (-1, 4) {B};
        \draw node[scale=0.7, anchor=south] at (0, 1) {D};

        
    \end{tikzpicture}

    }

    \newcommand{\BCDyellowleft}{
        \begin{tikzpicture}[scale=0.5]
        \coordinate (C) at (1, 0);
        \coordinate (D) at (2.5, -1);
        \coordinate (B) at (3, 0);

        \draw pic[fill=yellow, angle radius=1cm] {angle = D--C--B};
        

        \draw node[scale=0.7, anchor=north] at (1, 1) {C};
        \draw node[scale=0.7, anchor=south] at (3, 0) {B};
        \draw node[scale=0.7, anchor=south] at (3, -2) {D};

        
    \end{tikzpicture}

    }

    \newcommand{\BCDyellowright}{
        \begin{tikzpicture}[scale=0.5]
        \coordinate (C) at (1, 0);
        \coordinate (E) at (1.5, -1);
        \coordinate (B) at (3, 0);

        \draw pic[fill=yellow, angle radius=1cm] {angle = C--B--E};
        

        \draw node[scale=0.7, anchor=north] at (1, 1) {C};
        \draw node[scale=0.7, anchor=south] at (3, 0) {B};
        \draw node[scale=0.7, anchor=south] at (1, -2) {E};

        
    \end{tikzpicture}

    }

    \newcommand{\BCEyellowarealeft}{
        \begin{tikzpicture}[scale=0.5]
        \coordinate (C) at (1, 0);
        \coordinate (D) at (2.5, -1);
        \coordinate (B) at (3, 0);
        \coordinate (E) at (0, -2);


        \draw pic[fill=yellow, angle radius=1cm] {angle = D--C--B};
        \draw[line width=1mm, black] (1, 0) -- (0, -2);
        \draw[line width=1mm, yellow] (2.5, -1) -- (0, -2);

        \draw node[scale=0.7, anchor=north] at (1, 1) {C};
        \draw node[scale=0.7, anchor=south] at (3, 0) {B};
        \draw node[scale=0.7, anchor=south] at (0, -3) {E};

        
    \end{tikzpicture}

    }



    \newcommand{\BCDyellowarearight}{
        \begin{tikzpicture}[scale=0.5]
        \coordinate (C) at (1, 0);
        \coordinate (E) at (1.5, -1);
        \coordinate (B) at (3, 0);
        \coordinate(D) at (4, -2)

        \draw pic[fill=yellow, angle radius=1cm] {angle = C--B--E};
        \draw[line width=1mm, black] (3, 0) -- (4, -2);
        \draw[line width=1mm, yellow] (1.5, -1) -- (4, -2);
        

        \draw node[scale=0.7, anchor=north] at (1, 1) {C};
        \draw node[scale=0.7, anchor=south] at (3, 0) {B};
        \draw node[scale=0.7, anchor=south] at (4, -3) {D};

        
    \end{tikzpicture}

    }

    \newcommand{\BCEtrileft}{
        \begin{tikzpicture}[scale=0.5]
        \coordinate (C) at (1, 0);
        \coordinate (B) at (3, 0);
        \coordinate (E) at (0, -2);


        \draw[line width=1mm, black] (1, 0) -- (3, 0);
        \draw[line width=1mm, blue] (0, -2) -- (3, 0);
        \draw[line width=1mm, yellow] (0, -2) -- (1, 0);


        \draw node[scale=0.7, anchor=north] at (1, 1) {C};
        \draw node[scale=0.7, anchor=south] at (3, 0) {B};
        \draw node[scale=0.7, anchor=south] at (0, -3) {E};

        
    \end{tikzpicture}

    }

    
    \newcommand{\BCDtriright}{
        \begin{tikzpicture}[scale=0.5]
        \coordinate (C) at (1, 0);
        \coordinate (B) at (3, 0);
        \coordinate(D) at (4, -2)

        \draw[line width=1mm, black] (3, 0) -- (1, 0);
        \draw[line width=1mm, yellow] (3, 0) -- (4, -2);
        \draw[line width=1mm, blue] (1, 0) -- (4, -2);

        

        \draw node[scale=0.7, anchor=north] at (1, 1) {C};
        \draw node[scale=0.7, anchor=south] at (3, 0) {B};
        \draw node[scale=0.7, anchor=south] at (4, -3) {D};

        
    \end{tikzpicture}

    }

    \newcommand{\ABCtri}{
        \begin{tikzpicture}[scale=0.5]
        \coordinate (A) at (2, 2);
        \coordinate (B) at (3, 0);
        \coordinate(C) at (0, 0)

        \draw[line width=1mm, black] (3, 0) -- (1, 0);
        \draw[line width=1mm, red] (3, 0) -- (2, 2);
        \draw[line width=1mm, red] (1, 0) -- (2, 2);

        

        \draw node[scale=0.7, anchor=north] at (1, 0) {C};
        \draw node[scale=0.7, anchor=south] at (3, -1) {B};
        \draw node[scale=0.7, anchor=south] at (2, 2) {A};

        
    \end{tikzpicture}

    }
    





    
    \Large{
        \hspace{3cm} КНИГА I ПРЕДЛ. V. ТЕОРЕМА \hspace{3cm}\emph{29}
    }
    \\
    \begin{multicols}{2}
        
        \setlength{\columnsep}{-5cm}
        
        \begin{multicols}{2}
            \includegraphics[width=2.5cm, height=2.5cm ]{images/letter.png}
			
            \vfill\null
            \columnbreak
			
            \noindent глы \textit{при основании любого равнобедренного треугольника \ABCtri равны между собой и по продолжении равных сторон углы под основанием будут равны между собой.} 
            \columnbreak

        \end{multicols}
        \columnbreak
    \end{multicols}

        \begin{center}
            Продлим $\AB и \AC$ (пост. II),
            
            возьмем $\BD = \CE$ (пр. I.з);

            проведем $\BE и \CD.$

            Тогда в $\ABE и \ACD$

            получим $\AD = \EA$ (конст.)

            $\ABCblack$ общим обоим,

            и $\AB = \AC$ (гип.)

            $\therefore \ACDarea = \ABEarea, \BE = \CD;$

            и $\BCEblueleft = \BCDblueright$(пр.I.4).

            \vspace{5mm}
            
            Так же у $\BCEtrileft = \BCDtriright$

            получим $\BD = \CE$

            $\BCEblueleft = \BCDblueright$ и $\BE = \CD$

            $\therefore \BCEyellowarealeft = \BCDyellowarearight$  и $ \BCDyellowleft = \BCDyellowright$ (пр. I.4);

            но $\ACDarea = \ABEarea,  \therefore \ABCblueleft = \ABCblueright$ (акс. III).
            
            
        \end{center}
        
        \begin{flushright}ч.т.д.\end{flushright}
        
    
\end{document}